% \DeclareDocumentMetadata {lang=en-US}

% xmp metadata for pdf
% Originally used \usepackage[a-2a]{pdfx}
% \usepackage{hyperxmp} replaced it
% \RequirePackage{pdfmanagement-testphase} replaced it
% \PassOptionsToPackage{enable-debug,check-declarations}{expl3} broke with version 0.9 of tagpdf
% \ExplSyntaxOn no need for these 3 lines because metadata can handle it
% \pdfmanagement_add:nnn{Catalog}{Lang}{(enUS)} enUS is wrong, should be en-US
% \ExplSyntaxOff

\documentclass[11pt,
  english,
  letterpaper,
]{article}
\usepackage{sa4ss}
\usepackage{amsmath,amssymb,array}
\usepackage{booktabs}

% From tagged-template.latex
\usepackage{lmodern}
\usepackage{ifxetex,ifluatex}
\ifnum 0\ifxetex 1\fi\ifluatex 1\fi=0 % if pdftex
  \usepackage[T1]{fontenc}
  \usepackage[utf8]{inputenc}
  \usepackage{textcomp} % provide euro and other symbols
\else % if luatex or xetex
  \usepackage{unicode-math}
  \defaultfontfeatures{Scale=MatchLowercase}
  \defaultfontfeatures[\rmfamily]{Ligatures=TeX,Scale=1}
\fi

% Use upquote if available, for straight quotes in verbatim environments
\IfFileExists{upquote.sty}{\usepackage{upquote}}{}
\IfFileExists{microtype.sty}{% use microtype if available
  \usepackage[]{microtype}
  \UseMicrotypeSet[protrusion]{basicmath} % disable protrusion for tt fonts
}{}
\makeatletter
\@ifundefined{KOMAClassName}{% if non-KOMA class
  \IfFileExists{parskip.sty}{%
    \usepackage{parskip}
  }{% else
    \setlength{\parindent}{0pt}
    \setlength{\parskip}{6pt plus 2pt minus 1pt}}
}{% if KOMA class
  \KOMAoptions{parskip=half}}
\makeatother
\usepackage{xcolor}
\IfFileExists{xurl.sty}{\usepackage{xurl}}{} % add URL line breaks if available
\hypersetup{
  pdftitle={Status of Canary Rockfish (Sebastes pinniger) along the U.S. West coast in 2023},
  pdflang={en},
  hidelinks,
  pdfcreator={LaTeX via pandoc}}
\urlstyle{same} % disable monospaced font for URLs
\usepackage{longtable}
% Correct order of tables after \paragraph or \subparagraph
\usepackage{etoolbox}
\makeatletter
\patchcmd\longtable{\par}{\if@noskipsec\mbox{}\fi\par}{}{}
\makeatother
% Allow footnotes in longtable head/foot
\IfFileExists{footnotehyper.sty}{\usepackage{footnotehyper}}{\usepackage{footnote}}
\makesavenoteenv{longtable}
\usepackage{graphicx}
\makeatletter
\def\maxwidth{\ifdim\Gin@nat@width>\linewidth\linewidth\else\Gin@nat@width\fi}
\def\maxheight{\ifdim\Gin@nat@height>\textheight\textheight\else\Gin@nat@height\fi}
\makeatother
% Scale images if necessary, so that they will not overflow the page
% margins by default, and it is still possible to overwrite the defaults
% using explicit options in \includegraphics[width, height, ...]{}
\setkeys{Gin}{width=\maxwidth,height=\maxheight,keepaspectratio}
% Set default figure placement to htbp
\makeatletter
\def\fps@figure{htbp}
\makeatother
\setlength{\emergencystretch}{3em} % prevent overfull lines
\providecommand{\tightlist}{%
  \setlength{\itemsep}{0pt}\setlength{\parskip}{0pt}}
\setcounter{secnumdepth}{5}
\ifxetex
  % Load polyglossia as late as possible: uses bidi with RTL langages (e.g. Hebrew, Arabic)
  \usepackage{polyglossia}
  \setmainlanguage[]{}
\else
  \usepackage[shorthands=off,main=english]{babel}
\fi

%Define cslreferences environment, required by pandoc 2.8
%https://github.com/rstudio/rmarkdown/issues/1649
\newlength{\csllabelwidth}
\setlength{\csllabelwidth}{3em}
\newlength{\cslhangindent}
\setlength{\cslhangindent}{1.5em}
% for Pandoc 2.8 to 2.10.1
\newenvironment{cslreferences}%
  {}%
  {\par}
% For Pandoc 2.11+
\newenvironment{CSLReferences}[2] % #1 hanging-ident, #2 entry spacing
 {% don't indent paragraphs
  \setlength{\parindent}{0pt}
  % turn on hanging indent if param 1 is 1
  \ifodd #1 \everypar{\setlength{\hangindent}{\cslhangindent}}\ignorespaces\fi
  % set entry spacing
  \ifnum #2 > 0
  \setlength{\parskip}{#2\baselineskip}
  \fi
 }%
 {}
\usepackage{calc}  % for \widthof, \maxof in minipage
\newcommand{\CSLBlock}[1]{#1\hfill\break}
\newcommand{\CSLLeftMargin}[1]{\parbox[t]{\csllabelwidth}{#1}}
\newcommand{\CSLRightInline}[1]{\parbox[t]{\linewidth - \csllabelwidth}{#1}\break}
\newcommand{\CSLIndent}[1]{\hspace{\cslhangindent}#1}


\providecommand{\tightlist}{%
  \setlength{\itemsep}{0pt}\setlength{\parskip}{0pt}}


\date{}
\newcommand{\trTitle}{Status of Canary Rockfish (\emph{Sebastes pinniger}) along the U.S. West coast in 2023}
\newcommand{\trYear}{2023}
\newcommand{\trMonth}{January}
\newcommand{\trAuthsLong}{truetrue}
\newcommand{\trAuthsBack}{Langseth, B.J., K.L. Oken}
\newcommand{\trCitation}{
\begin{hangparas}{1em}{1}
\trAuthsBack{}. \trYear{}. \trTitle{}. \glsentrylong{pfmc}, Portland, Oregon. \pageref{LastPage}{}\,p.
\end{hangparas}}

\begin{document}

%%%%% Frontmatter %%%%%

% Footnote symbols in front matter
\renewcommand*{\thefootnote}{\fnsymbol{footnote}}

\small
\thispagestyle{empty}
\pagenumbering{roman}
\noindent
\begin{center}
\title{Status of Canary Rockfish (\emph{Sebastes pinniger}) along the U.S. West coast in 2023}
% \textnormal{\MakeTextUppercase{\trTitle{}}}
\vspace{1.5cm}
{\Large\textbf\newline{Status of Canary Rockfish (\emph{Sebastes pinniger}) along the U.S. West coast in 2023}}
\vfill
by\\
Brian J. Langseth\textsuperscript{1}\\
Kiva L. Oken\textsuperscript{1}\vfill
\textsuperscript{1}Northwest Fisheries Science Center, U.S. Department of Commerce, National Oceanic and Atmospheric Administration, National Marine Fisheries Service, 2725 Montlake Boulevard East, Seattle, Washington 98112\vfill
\trMonth{} \trYear{}
\end{center}
\clearpage

% Fourth page: Colophon
\thispagestyle{empty}
\vspace*{\fill}
\begin{center}
\copyright{} \glsentrylong{pfmc}, \trYear{}\\
\end{center}
\par
\bigskip
\noindent
Correct citation for this publication:
\bigskip
\par
\trCitation{}
\clearpage

% Add TOC to pdf bookmarks (clickable pdf)
\pdfbookmark[1]{\contentsname}{toc}

% Table of contents page, lists of figures and tables
\tableofcontents\clearpage
\label{TRlastRoman}
\clearpage

% Table of contents
\newpage
\thispagestyle{empty} % to remove page number

% Settings for the main document
\pagenumbering{arabic}  % Regular page numbers
\pagestyle{plain}  % No page number on first page of main document, use 'empty'
\renewcommand*{\thefootnote}{\arabic{footnote}}  % Back to numeric footnotes
\setcounter{footnote}{0}  % And start at 1
\renewcommand{\headrulewidth}{0.5pt}
\renewcommand{\footrulewidth}{0.5pt}
%\pagestyle{fancy}\fancyhead[c]{Draft: Do not cite or circulate}

\newcommand{\lt}{\ensuremath <}
\newcommand{\gt}{\ensuremath >}

\pagebreak
\pagenumbering{roman}
\setcounter{page}{1}

\renewcommand{\thetable}{\roman{table}}
\renewcommand{\thefigure}{\roman{figure}}

\setlength\parskip{0.5em plus 0.1em minus 0.2em}

\hypertarget{executive-summary}{%
\section*{Executive summary}\label{executive-summary}}
\addcontentsline{toc}{section}{Executive summary}

\hypertarget{stock}{%
\subsection*{Stock}\label{stock}}
\addcontentsline{toc}{subsection}{Stock}

This assessment reports the status of Canary Rockfish (\emph{Sebastes pinniger}) off the U.S. West coast using data through xxxx.

\hypertarget{catches}{%
\subsection*{Catches}\label{catches}}
\addcontentsline{toc}{subsection}{Catches}

Replace text with trends and current levels. Include Table for last 10 years. Include Figure with long-term estimates.

\hypertarget{data-and-assessment}{%
\subsection*{Data and assessment}\label{data-and-assessment}}
\addcontentsline{toc}{subsection}{Data and assessment}

This assessment uses the stock assessment framework Stock Synthesis

(SS3).

Replace text with date of last assessment, type of assessment model, data available, new information, and information lacking.

\hypertarget{stock-biomass-and-dynamics}{%
\subsection*{Stock biomass and dynamics}\label{stock-biomass-and-dynamics}}
\addcontentsline{toc}{subsection}{Stock biomass and dynamics}

Replace text with trends and current levels relative to virgin or historic levels and description of uncertainty. Include Table for last 10 years. Include Figure with long-term estimates.

\hypertarget{recruitment}{%
\subsection*{Recruitment}\label{recruitment}}
\addcontentsline{toc}{subsection}{Recruitment}

Replace text with trends and current levels relative to virgin or historic levels and description of uncertainty. Include Table for last 10 years. Include Figure with long-term estimates.

\hypertarget{exploitation-status}{%
\subsection*{Exploitation status}\label{exploitation-status}}
\addcontentsline{toc}{subsection}{Exploitation status}

Replace text with total catch divided by exploitable biomass or SPR harvest rate. Include Table for last 10 years. Include Figure with trend in f relative to target vs.~trend in biomass relative to the target.

\hypertarget{ecosystem-considerations}{%
\subsection*{Ecosystem considerations}\label{ecosystem-considerations}}
\addcontentsline{toc}{subsection}{Ecosystem considerations}

Replace text with a summary of reviewed environmental and ecosystem factors that appear to be correlated with stock dynamics. These may include variability in they physical environment, habitat, competitors, prey, or predators that directly or indirectly affects the stock's status, vital rates (growth, survival, productivity/recruitment) or range and distribution. Note which, if any, ecosystem factors are used in the assessment and how (e.g., as background information, in data preparations, as data inputs, in decisions about model structure).

\hypertarget{reference-points}\), i.e., the \(B_{MSY}\) proxy and the equilibrium stock size that results from fishing at the default harvest rate, i.e., the \(F_{MSY}\) proxy. Include Table of estimated reference points for ssb, SPR, exploitation rate, and yield based on SSB proxy for MSY, SPR proxy for MSY, and estimated MSY values.

\hypertarget{management-performance}{%
\subsection*{Management performance}\label{management-performance}}
\addcontentsline{toc}{subsection}{Management performance}

Include Table of most recent 10 years of catches in comparison with OFL, ABC, HG, and OY/ACL values, overfishing levels, actual catch and discard. Include OFL (encountered), OFL (retained), and OFL (dead) if different due to discard and discard mortality.

\hypertarget{unresolved-problems-and-major-uncertainties}{%
\subsection*{Unresolved problems and major uncertainties}\label{unresolved-problems-and-major-uncertainties}}
\addcontentsline{toc}{subsection}{Unresolved problems and major uncertainties}

Replace text with any special issues that complicate scientific assessment, questions about the best model scenario, etc.

\hypertarget{decision-table-and-projections}{%
\subsection*{Decision table and projections}\label{decision-table-and-projections}}
\addcontentsline{toc}{subsection}{Decision table and projections}

Replace text with projected yields (OFL, ABC, and ACL), spawning biomass, and stock depletion levels for each year. OFL calculations should be based on the assumption that future catches equal ABCs and not OFLs.

\hypertarget{scientific-uncertainty}{%
\subsection*{Scientific uncertainty}\label{scientific-uncertainty}}
\addcontentsline{toc}{subsection}{Scientific uncertainty}

Replace text with the sigma value and the basis for its calculation.

\hypertarget{research-and-data-needs}{%
\subsection*{Research and data needs}\label{research-and-data-needs}}
\addcontentsline{toc}{subsection}{Research and data needs}

Replace text with information gaps that seriously impede the stock assessment.

\pagebreak
\setlength{\parskip}{5mm plus1mm minus1mm}
\pagenumbering{arabic}
\setcounter{page}{1}
\renewcommand{\thefigure}{\arabic{figure}}
\renewcommand{\thetable}{\arabic{table}}
\setcounter{table}{0}
\setcounter{figure}{0}

\hypertarget{introduction}{%
\section{Introduction}\label{introduction}}

\hypertarget{basic-information}{%
\subsection{Basic Information}\label{basic-information}}

This assessment reports the status of Canary Rockfish (\emph{Sebastes pinniger}) off the U.S. West coast using data through xxxx.

\hypertarget{life-history}{%
\subsection{Life History}\label{life-history}}

Replace text.

\hypertarget{ecosystem-considerations-1}{%
\subsection{Ecosystem Considerations}\label{ecosystem-considerations-1}}

Replace text.

\hypertarget{historical-and-current-fishery-information}{%
\subsection{Historical and Current Fishery Information}\label{historical-and-current-fishery-information}}

Replace text.

\hypertarget{summary-of-management-history-and-performance}{%
\subsection{Summary of Management History and Performance}\label{summary-of-management-history-and-performance}}

Replace text.

\hypertarget{foreign-fisheries}{%
\subsection{Foreign Fisheries}\label{foreign-fisheries}}

Replace text.

\hypertarget{data}{%
\section{Data}\label{data}}

Data comprise the foundational components of stock assessment models. The decision to include or exclude particular data sources in an assessment model depends on many factors. These factors often include, but are not limited to, the way in which data were collected (e.g., measurement method and consistency); the spatial and temporal coverage of the data; the quantity of data available per desired sampling unit; the representativeness of the data to inform the modeled processes of importance; timing of when the data were provided; limitations imposed by the Terms of Reference; and the presence of an avenue for the inclusion of the data in the assessment model. Attributes associated with a data source can change through time, as can the applicability of the data source when different modeling approaches are explored (e.g., stock structure or time-varying processes). Therefore, the specific data sources included or excluded from this assessment should not necessarily constrain the selection of data sources applicable to future stock assessments for Canary Rockfish. Even if a data source is not directly used in the stock assessment they can provide valuable insights into biology, fishery behavior, or localized dynamics.

Data from a wide range of programs were available for possible inclusion in the current assessment model. Descriptions of each data source included in the model (Figure \ref{fig:data-plot}) and sources that were explored but not included in the base model are provided below. Data that were excluded from the base model were explicitly explored during the development of this stock assessment or have not changed since their past exploration in a previous Canary Rockfish stock assessment. In some cases, the inclusion of excluded data sources were explored through sensitivity analyses (see Section \ref{assessment-model}).

\hypertarget{fishery-dependent-data}{%
\subsection{Fishery-Dependent Data}\label{fishery-dependent-data}}

\hypertarget{fishery-independent-data}{%
\subsection{Fishery-Independent Data}\label{fishery-independent-data}}

\hypertarget{section}{%
\subsubsection{\texorpdfstring{\acrlong{s-aslope}}{}}\label{section}}

The \gls{s-aslope} operated during the months of October to November aboard the R/V \emph{Miller Freeman}. Partial survey coverage of the US west coast occurred during the years 1988-1996 and complete coverage (north of 34\textdegree 30\textquotesingle S) during the years 1997 and 1999-2001. Typically, only these four years that are seen as complete surveys are included in assessments.

\hypertarget{section-1}{%
\subsubsection{\texorpdfstring{\acrlong{s-ccfrp}}{}}\label{section-1}}

Since 2007, the \gls{s-ccfrp} has monitored several areas in California to evaluate the performance of \glspl{mpa} and understand nearshore fish populations (Wendt and Starr 2009; Starr et al. 2015). In 2017, the survey expanded beyond the four \Gls{mpa}s in central California (Año Nuevo, Point Lobos, Point Buchon, and Piedras Blancas) to include the entire California coast. Fish are collected by volunteer anglers aboard \glspl{cpfv} guided by one of the following academic institutions based on proximity to fishing location: Humboldt State University; Bodega Marine Laboratories; Moss Landing Marine Laboratories; Cal Poly San Luis Obispo; University of California, Santa Barbara; and Scripps Institution of Oceanography.

Surveys consist of fishing with hook-and-line gear for 30-45 minutes within randomly chosen 500 by 500 m grid cells within and outside \glspl{mpa}. Prior to 2017, all fish were measured for length and release or descended to depth; since then, some were sampled for otoliths and fin clips.

\hypertarget{section-2}{%
\subsubsection{\texorpdfstring{\acrlong{s-tri}}{}}\label{section-2}}

The \gls{s-tri} was first conducted by the \gls{afsc} in 1977, and the survey continued until 2004 (Weinberg et al. 2002). Its basic design was a series of equally-spaced east-to-west transects across the continential shelf from which searches for tows in a specific depth range were initiated. The survey design changed slightly over time. In general, all of the surveys were conducted in the mid summer through early fall. The 1977 survey was conducted from early July through late September. The surveys from 1980 through 1989 were conducted from mid-July to late September. The 1992 survey was conducted from mid July through early October. The 1995 survey was conducted from early June through late August. The 1998 survey was conducted from early June through early August. Finally, the 2001 and 2004 surveys were conducted from May to July.

Haul depths ranged from 91-457 m during the 1977 survey with no hauls shallower than 91 m. Due to haul performance issues and truncated sampling with respect to depth, the data from 1977 were omitted from this analysis. The surveys in 1980, 1983, and 1986 covered the US West Coast south to 36.8\textdegree N latitude and a depth range of 55-366 m. The surveys in 1989 and 1992 covered the same depth range but extended the southern range to 34.5\textdegree N (near Point Conception). From 1995 through 2004, the surveys covered the depth range 55-500 m and surveyed south to 34.5\textdegree N. In 2004, the final year of the \gls{s-tri} series, the \gls{nwfsc} \gls{fram} conducted the survey following similar protocols to earlier years.

\hypertarget{section-3}{%
\subsubsection{\texorpdfstring{\acrlong{s-wcgbt}}{}}\label{section-3}}

The \Gls{s-wcgbt} is based on a random-grid design; covering the coastal waters from a depth of 55-1,280 m (Bradburn, Keller, and Horness 2011). This design generally uses four industry-chartered vessels per year assigned to a roughly equal number of randomly selected grid cells and divided into two `passes' of the coast. Two vessels fish from north to south during each pass between late May to early October. This design therefore incorporates both vessel-to-vessel differences in catchability, as well as variance associated with selecting a relatively small number (approximately 700) of possible cells from a very large set of possible cells spread from the Mexican to the Canadian borders.

\hypertarget{biological-data}{%
\subsection{Biological Data}\label{biological-data}}

\hypertarget{natural-mortality}{%
\subsubsection{Natural Mortality}\label{natural-mortality}}

\hypertarget{maturation-and-fecundity}{%
\subsubsection{Maturation and Fecundity}\label{maturation-and-fecundity}}

\hypertarget{sex-ratio}{%
\subsubsection{Sex Ratio}\label{sex-ratio}}

\hypertarget{length-weight-relationship}{%
\subsubsection{Length-Weight Relationship}\label{length-weight-relationship}}

\hypertarget{growth-length-at-age}{%
\subsubsection{Growth (Length-at-Age)}\label{growth-length-at-age}}

\hypertarget{ageing-precision-and-bias}{%
\subsubsection{Ageing Precision and Bias}\label{ageing-precision-and-bias}}

\hypertarget{environmental-and-ecosystem-data}{%
\subsection{Environmental and Ecosystem Data}\label{environmental-and-ecosystem-data}}

\hypertarget{assessment-model}{%
\section{Assessment Model}\label{assessment-model}}

\hypertarget{summary-of-previous-assessments-and-reviews}{%
\subsection{Summary of Previous Assessments and Reviews}\label{summary-of-previous-assessments-and-reviews}}

\hypertarget{history-of-modeling-approaches-not-required-for-an-update-assessment}{%
\subsubsection{History of Modeling Approaches (not required for an update assessment)}\label{history-of-modeling-approaches-not-required-for-an-update-assessment}}

\hypertarget{most-recent-star-panel-and-ssc-recommendations-not-required-for-an-update-assessment}{%
\subsubsection{Most Recent STAR Panel and SSC Recommendations (not required for an update assessment)}\label{most-recent-star-panel-and-ssc-recommendations-not-required-for-an-update-assessment}}

\hypertarget{response-to-groundfish-subcommittee-requests-not-required-in-draft}{%
\subsubsection{Response to Groundfish Subcommittee Requests (not required in draft)}\label{response-to-groundfish-subcommittee-requests-not-required-in-draft}}

\hypertarget{model-structure-and-assumptions}{%
\subsection{Model Structure and Assumptions}\label{model-structure-and-assumptions}}

\hypertarget{model-changes-from-the-last-assessment-not-required-for-an-update-assessment}{%
\subsubsection{Model Changes from the Last Assessment (not required for an update assessment)}\label{model-changes-from-the-last-assessment-not-required-for-an-update-assessment}}

\hypertarget{modeling-platform-and-structure}{%
\subsubsection{Modeling Platform and Structure}\label{modeling-platform-and-structure}}

General model specifications (e.g., executable version, model structure, definition of fleets and areas)

\hypertarget{model-parameters}{%
\subsubsection{Model Parameters}\label{model-parameters}}

Describe estimated vs.~fixed parameters, priors

\hypertarget{key-assumptions-and-structural-choices}{%
\subsubsection{Key Assumptions and Structural Choices}\label{key-assumptions-and-structural-choices}}

\hypertarget{base-model-results}{%
\subsection{Base Model Results}\label{base-model-results}}

\hypertarget{parameter-estimates}{%
\subsubsection{Parameter Estimates}\label{parameter-estimates}}

\hypertarget{fits-to-the-data}{%
\subsubsection{Fits to the Data}\label{fits-to-the-data}}

\hypertarget{population-trajectory}{%
\subsubsection{Population Trajectory}\label{population-trajectory}}

\hypertarget{reference-points-1}{%
\subsubsection{Reference Points}\label{reference-points-1}}

\hypertarget{model-diagnostics}{%
\subsection{Model Diagnostics}\label{model-diagnostics}}

Describe all diagnostics

\hypertarget{convergence}{%
\subsubsection{Convergence}\label{convergence}}

\hypertarget{sensitivity-analyses}{%
\subsubsection{Sensitivity Analyses}\label{sensitivity-analyses}}

\hypertarget{retrospective-analysis}{%
\subsubsection{Retrospective Analysis}\label{retrospective-analysis}}

\hypertarget{likelihood-profiles}{%
\subsubsection{Likelihood Profiles}\label{likelihood-profiles}}

\hypertarget{unresolved-problems-and-major-uncertainties-1}{%
\subsubsection{Unresolved Problems and Major Uncertainties}\label{unresolved-problems-and-major-uncertainties-1}}

\hypertarget{management}{%
\section{Management}\label{management}}

\hypertarget{reference-points-2}{%
\subsection{Reference Points}\label{reference-points-2}}

\hypertarget{unresolved-problems-and-major-uncertainties-2}{%
\subsection{Unresolved Problems and Major Uncertainties}\label{unresolved-problems-and-major-uncertainties-2}}

\hypertarget{harvest-projections-and-decision-tables}{%
\subsection{Harvest Projections and Decision Tables}\label{harvest-projections-and-decision-tables}}

\hypertarget{evaluation-of-scientific-uncertainty}{%
\subsection{Evaluation of Scientific Uncertainty}\label{evaluation-of-scientific-uncertainty}}

\hypertarget{research-and-data-needs-1}{%
\subsection{Research and Data Needs}\label{research-and-data-needs-1}}

\hypertarget{acknowledgments}{%
\section{Acknowledgments}\label{acknowledgments}}

Here are all the mad props!

\clearpage

\hypertarget{references}{%
\section{References}\label{references}}

\hypertarget{refs}{}
\begin{CSLReferences}{1}{0}
\leavevmode\vadjust pre{\hypertarget{ref-bradburn_2003_2011}{}}%
Bradburn, M. J., A. A Keller, and B. H. Horness. 2011. {``The 2003 to 2008 {US} {West} {Coast} Bottom Trawl Surveys of Groundfish Resources Off {Washington}, {Oregon}, and {California}: Estimates of Distribution, Abundance, Length, and Age Composition.''} US Department of Commerce, National Oceanic; Atmospheric Administration, National Marine Fisheries Service.

\leavevmode\vadjust pre{\hypertarget{ref-Starr2015}{}}%
Starr, R. M., D. E. Wendt, C. L. Barnes, C. I. Marks, D. Malone, G. Waltz, K. T. Schmidt, et al. 2015. {``Variation in Responses of Fishes Across Multiple Reserves Within a Network of Marine Protected Areas in Temperate Waters.''} \emph{PLoS One2} 10 (3): p.e0118502.

\leavevmode\vadjust pre{\hypertarget{ref-weinberg_2001_2002}{}}%
Weinberg, K. L., M. E. Wilkins, F. R. Shaw, and M. Zimmermann. 2002. {``The 2001 {Pacific} {West} {Coast} Bottom Trawl Survey of Groundfish Resources: Estimates of Distribution, Abundance and Length and Age Composition.''} \{NOAA\} \{Technical\} \{Memorandum\} NMFS-AFSC-128. U.S. Department of Commerce.

\leavevmode\vadjust pre{\hypertarget{ref-Wendt2009}{}}%
Wendt, D. E., and R. M. Starr. 2009. {``Collaborative Research: An Effective Way to Collect Data for Stock Assessments and Evaluate Marine Protected Areas in {C}alifornia.''} \emph{Marine and Coastal Fisheries: Dynamics, Management, and Ecosystem Science.} 1: 315--24.

\end{CSLReferences}

\clearpage

\hypertarget{tables}{%
\section{Tables}\label{tables}}

\clearpage

\hypertarget{figures}{%
\section{Figures}\label{figures}}

\begin{figure}
\centering
\includegraphics[width=1\textwidth,height=1\textheight]{data-plot.png}
\caption{Summary of data sources used in the base model.\label{fig:data-plot}}
\end{figure}
\end{document}
